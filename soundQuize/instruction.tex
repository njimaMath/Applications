\documentclass{article}
\usepackage{amsmath}
\usepackage{amssymb}
\usepackage{graphicx}
\usepackage[T1]{fontenc}
\usepackage{hyperref}

\begin{document}

\title{Instructions for Building a Multi-Category English Listening Quiz App\\
Using Mozilla Common Voice from Hugging Face Hub}
\author{Your Name}
\date{\today}
\maketitle

\section*{Purpose of the Application}

The purpose of this application is to help English learners improve their ability to distinguish confusing sounds in real speech.
In the app, the user hears a sentence (audio), and on the screen the same sentence appears with one word hidden, for example
\[
\text{I \fbox{\phantom{have}} a pen.}
\]
Below the sentence, the app shows two confusing choices, such as
\[
\text{have} \quad\text{and}\quad \text{haven't}.
\]
The user selects one of them, and the app immediately shows whether the answer is correct.
The same mechanism is used for other pairs such as L/R, V/B, S/TH, and D/TH.

\section*{Step 1: Download Data from Hugging Face Hub}

The Mozilla Common Voice dataset is hosted on the Hugging Face Hub, a platform for sharing datasets and models. You can download it programmatically.

\subsection*{1.1 Choose the dataset and language}

\begin{itemize}
    \item The main dataset is located at \href{https://huggingface.co/datasets/mozilla-foundation/common_voice_16_1}{mozilla-foundation/common\_voice\_16\_1}.
    \item You will need to choose a language (e.g., \texttt{en} for English) and a data split (e.g., \texttt{validated}, \texttt{train}).
\end{itemize}

\subsection*{1.2 Download the dataset}

Use the provided Python script to download the data. From the command line, run a command of the following form (replace placeholders with your desired values):

\begin{verbatim}
python download_common_voice.py --language en --split validated
\end{verbatim}

This will download the specified part of the dataset and save it as a \texttt{.tar.gz} archive in the \texttt{data/} directory, then extract it to \texttt{data/corpus/}.

\subsection*{1.3 Extract and prepare quiz items}

After downloading and extracting the archive, you will have:

\begin{itemize}
  \item audio files (for example, \verb|.mp3|),
  \item metadata files such as \verb|train.tsv|, \verb|dev.tsv|, \verb|test.tsv|.
\end{itemize}

Write a small script (with help from your code assistant) that:

\begin{itemize}
  \item reads a metadata file (for example, \verb|train.tsv|),
  \item selects English sentences that contain target words for your categories,
  \item records, for each selected sentence:
        \begin{itemize}
          \item the audio file path,
          \item the full transcript,
          \item the target word (to be blanked),
          \item a confusing partner word,
          \item the category (positive/negative, L/R, V/B, S/TH, D/TH).
        \end{itemize}
\end{itemize}

Store the result in a simple JSON file, for example:

\begin{verbatim}
[
  {
    "category": "positive_negative",
    "audio": "clips/abc123.mp3",
    "sentence": "I have a pen.",
    "blank_index": 2,
    "optionA": "have",
    "optionB": "haven't",
    "correct": "have"
  },
  {
    "category": "L_R",
    "audio": "clips/xyz789.mp3",
    "sentence": "Turn on the right light.",
    "blank_index": 4,
    "optionA": "light",
    "optionB": "right",
    "correct": "right"
  }
]
\end{verbatim}

\section*{Step 2: Set Up Your Coding Environment with a CLI Assistant}

Prepare a development environment where you can run Python from the command line. Then:

\begin{itemize}
  \item install Python 3 and common packages such as \verb|requests|, \verb|pydub| or \verb|playsound|,
  \item install your preferred AI coding assistant or CLI interface,
  \item configure your API key for that coding assistant.
\end{itemize}

You will use the assistant mainly to generate boilerplate code: loading JSON, playing audio, handling user input, and running quiz loops.

\section*{Step 3: Design the Quiz Flow}

Use your coding assistant to help you write a Python script with the following structure.

\subsection*{3.1 Category selection}

When the script starts:

\begin{itemize}
  \item load the JSON file containing quiz items,
  \item scan available categories in the data:
        \begin{itemize}
          \item positive\_negative
          \item L\_R
          \item V\_B
          \item S\_TH
          \item D\_TH
        \end{itemize}
  \item show a menu and let the user choose one category.
\end{itemize}

\subsection*{3.2 Single quiz round}

For each round in the chosen category, the script should:

\begin{enumerate}
  \item Randomly select one quiz item from that category.
  \item Play the corresponding audio sentence:
        \begin{itemize}
          \item for example, using \verb|playsound(item["audio"])|.
        \end{itemize}
  \item Display the sentence text on screen with the target word hidden.
        For instance, if the sentence is ``I have a pen.'' and the target word index is 2, print something like
        \[
        \text{I \fbox{\phantom{have}} a pen.}
        \]
        (In practice, you can construct this string in Python by replacing the target word by \verb|[   ]|.)
  \item Display two word choices directly under the sentence, for example:
        \[
        \text{A) have \quad B) haven't}
        \]
  \item Let the user type A or B (or click, if you use a web interface).
  \item Compare the choice with the correct answer and print feedback, for example
        ``Correct: the sentence was \textit{I have a pen.}'' or
        ``Incorrect: the correct answer was \textit{have}.''
  \item Optionally, show the full sentence with the correct word highlighted.
\end{enumerate}

Ask your coding assistant to generate helper functions such as:

\begin{itemize}
  \item \verb|load_quiz_data(filepath)| to read the JSON file,
  \item \verb|get_random_item(category, data)| to choose a random quiz item,
  \item \verb|play_audio(path)| to play the sound file,
  \item \verb|run_quiz_round(item)| to handle one question and feedback,
  \item \verb|run_quiz_loop(category, data)| to repeat several rounds.
\end{itemize}

\section*{Step 4: Implement a Simple Interface}

\subsection*{4.1 Terminal interface}

For a terminal-based version:

\begin{itemize}
  \item use \verb|playsound| or a similar library to play audio,
  \item print the sentence with the blank and the two choices,
  \item read user input from the keyboard (\verb|input()| in Python),
  \item keep track of the score and number of questions.
\end{itemize}

\subsection*{4.2 Web interface (optional)}

For a small web app (for example, with Flask):

\begin{itemize}
  \item create a route like \verb|/quiz| that:
        \begin{itemize}
          \item chooses a quiz item,
          \item renders an HTML page with:
                \begin{itemize}
                  \item an \verb|<audio controls>| tag pointing to the audio file,
                  \item the sentence with one word replaced by a box,
                  \item two buttons or radio buttons for the two choices.
                \end{itemize}
        \end{itemize}
  \item when a button is pressed, send the answer back to the server and return a page with feedback and the full correct sentence.
\end{itemize}

You can ask your coding assistant to generate template Flask code and simple HTML pages.

\section*{Step 5: Test and Refine}

Finally:

\begin{itemize}
  \item test each category and adjust sentences so that the difficulty feels appropriate,
  \item verify that the audio always matches the text,
  \item gradually add more quiz items from Common Voice to expand your practice library.
\end{itemize}

\end{document}
